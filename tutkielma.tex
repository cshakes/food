\documentclass[12pt]{scrreprt}
\usepackage[finnish]{babel}
\usepackage[utf8]{inputenc}
\usepackage{amsmath}
\usepackage[toc,page]{appendix}

\renewcommand\emph{\textbf}
\addto\captionsfinnish{
  \renewcommand{\bibname}{Lähteet}
  \renewcommand{\appendixname}{Liitteet}
  \renewcommand{\appendixpagename}{Liitteet}
}
\renewcommand\appendixtocname{Liitteet}

\title{Mulla on nälkä!}
\subtitle{Ruoan tuotanto Somaliassa hörpäkkeiden funktiona}
\author{Jaakko Hannikainen \and Cliona Shakespeare}
\date{}
\publishers{Valkeakosken Tietotien lukio / Päivölän Kansanopisto\\
            Tieteenala: Mallintaminen}
\begin{document}
  \maketitle

  \begin{abstract}
    Tässä tutkitaan ruokaa Somaliassa ja silleen.
  \end{abstract}

  \tableofcontents

  \chapter{Johdanto}

  % Mitä: Somalia on kehitysmaa, ongelmia: ruokaa ei ole tarpeeksi jne,
  % ongelmien mallintamiseksi tämä simulaatio
  
  Somalia on afrikkalainen kehitysmaa, jolla ei ole ollut 1990-luvun jälkeen
  toimivaa hallitusta. Eräs sen lukuisista ongelmista on se, että ruokaa ei ole
  tarpeeksi kaikille, johtaen nälänhätään, kulkutautien leviämiseen, ja ihmisten
  kuolemiseen. Tutkimuskysymyksemme on, kuinka paljon ruokaa pelloilta tulee
  tietyissä olosuhteissa.
  
  Tämä malli on toteutettu osana mallikokonaisuutta, jossa mallinnetaan
  humanitaarisen avustuksen vaikutusta Somalian tilanteeseen. Ruoan tuotantoon
  liittyvät ilmasto-, vesi-, logistiikka- ja karttamallit.
  
  Esimerkki mallin käytöstä: avustusjärjestö haluaa tietää, miten se saisi
  autettua ihmisiä mahdollisimman tehokkaasti rajatuilla resursseilla. Järjestö
  pyörittää simulaatiota usein alkuarvoin, ja katsoo, millaisia tuloksia tulee
  milläkin alkuarvoilla. Järjestö voi käyttää näitä tietoja hyväkseen
  päätöksenteossaan.
  
  % Ruokaa ei tarpeeksi → nälänhätä → kulkutauteja → ihmisiä kuolee → Ongelma.
  
  % Miksi ei karjataloutta tms
  
  % Miksi: tulevaisuuden mallintaminen ja ennustaminen (Somalia! Suomen
  % nälkävuodet!), humanitaarisen avustuksen vaikutukset, ihmiskärsimyksen
  % vähentäminen 
  
  % Ilmiön idea (Somaliassa kasvatetaan ruokaa. Tätä pitää mallintaa.)
  
  % tulevaisuuden ennustaminen on kivaa – miksi?
  
  \chapter{Aineisto ja menetelmät}

  \section{Ilmiö}

  Maatalous on järjestelmällistä kasvin kasvattamista suurella mittakaavalla.

  % maatalous yleisesti ^

  Roosa Räty on tutkinut, mitä Somaliassa syödään. [viittaus!] Käytimme hänen
  tuloksiaan valitessamme kasvilajikkeita tutkittavaksi.
  
  Kasvien kasvamiseen vaikuttavat muun muassa sääolosuhteet (sadanta, lämpötila,
  auringonvalon määrä) ja maaperän pH. Nämä tekijät olemme huomioineet
  mallissamme. Jokaisella on optimiarvo, josta poiketessa sadon määrä laskee.

  Olemme todenneet mitättömiksi ja jättäneet huomiotta muun muassa lehtien
  pinta-alan, haihtumisen, juurien tiheyde, leveyspiirin ja tuulen vaikutukset
  satoon. Keskitymme myös maatalouteen, joten olemme jättäneet Somalian
  karjatalouden huomioimatta.
  
  % on-off: rankkasateet, tulvat, 
  
  \section{Malli}

  Malli ottaa alkuarvoikseen sademäärän, lämpötilan, auringonvalon määrän ja
  maaperän pH:n. Jokainen muunnetaan prosenttikertoimeksi kasvatetun kasvin
  mukaan. Sadon määrä saadaan kaavalla

  \begin{em}
  $ \text{sato} = (\text{pinta-ala}) \times (\text{teoreettinen maksimisato per
  alueyksikko}) \times (\text{viikkokerrointen keskiarvo}) $\end{em}
  jossa jokainen viikkokerroin saadaan kaavalla

  \begin{em}
  $ \text{viikkokerroin} = (\text{vesikerroin}) \times (\text{lämpökerroin})
  \times (\text{valokerroin}) \times (\text{pHkerroin}) $\end{em}

  % funktioiden kuvaus

  % Lyhyt kuvaus jokaisesta mahdollisesta viljelyskasvista; maissi, vehnä, durra

  % Perustelut juuri näille parametreille, kuvaus parametreistä, kuvaus,
  % merkitys, mahdolliset arvot, yksikkö → teoreettiset minimi- ja maksimiarvot
  % sadolle

  \subsection{Maissi}

  Maissi on kasvi, joka tuottaa yleensä keltaisia siemeniä. Siemenet ovat
  syötävä osa, loput kasvista jätetään syömättä. Maissi tarvitsee kostean ja
  ravinteikkaan maaperän. Sille ideaali vesimäärä olisi \cite{cropwater} mukaan
  keskimäärin 0.29 cm/vrk; annoimme parametrillemme vaihteluvälin 0.145 cm/vrk
  – 0.425 cm/vrk. \cite{ugandamaize} ja \cite{plessismaize} antavat maissille
  eri kasvulämpötiloja (5$^{\circ}$C – 42$^{\circ}$C ja 10$^{\circ}$C –
  32$^{\circ}$C). Mallinnuksen helpottamiseksi olemme valinneet
  lämpötilajakaumaksi 20$^{\circ}$C – 40$^{\circ}$C, optimi 30$^{\circ}$C.
  Maissille sopiva pH on välillä 5.5 – 7.0 \cite{corngrowing}. Käytämme
  ideaalina näiden keskiarvoa 6.25. Auringonvalolle arvioimme optimiksi
  8h/päivä, jakauma 6-10 tuntia päivässä. Maissi kasvaa 18 viikossa.

  \subsection{Vehnä}

  Vehnä on Suomessakin kasvava vilja. Sille ideaalilämpötila olisi 19$^{\circ}$C
  - 25$^{\circ}$C \cite{wheat}, valitsimme optimiksemme 22$^{\circ}$C. Käytämme
  pH:n jakaumana 6-7:ää \cite{wheatfert}, optimi 6,5. Veden ideaalimäärä olisi
  0.31 cm/vrk \cite{cropwater}, jakauma  0.155 – 0.465 cm/vrk. \cite{growwheat}
  mukaan vehnä tarvitsisi noin 6 tuntia auringonvaloa päivässä. Vehnä kasvaa
  noin 15 viikossa. citation needed

  \subsection{Durra}

  vesi 0.38 cm/vrk \cite{cropwater}.

  kasvaa noin 18 viikossa

  \chapter{Tulos}

  \section{Mallin käyttö simulaatiossa}

  Jotain, jotain, koodi, malli -$>$ koodiprosessi + koodin kuvailu

  \section{Testiajot}

  kauniita kuvia

  \chapter{Johtopäätökset}

  malli hyvä, koska
  
  paitsi että täyttä kuraa (kesken, yksinkertaistettu), asiat A-Ö, 1-9 ja
  loputkin huonosti parametroitu tai jätetty huomiotta jne; karjatalous!
  tarvitsee enemmän/parempaa lähtödataa
  
  siitä huolimatta: auttaa ymmärtämistä, osa kokonaisuutta, toimii ihan OK

  \begin{thebibliography}{9}

    \bibitem{cropwater}
    Wright \& Killer.
    \emph{Average water use for multiple crops}.
    2002.
    Haettu \today. \\
    http://www.extension.umn.edu/distribution/cropsystems/components/DC1322a.pdf

    \bibitem{ugandamaize}
    Ministry of Agriculture, Uganda.
    \emph{Maize production}.

    \bibitem{plessismaize}
    Jéan du Plessis.
    \emph{Maize production}.
    Haettu \today. \\
    http://www.arc.agric.za/arc-gci/Fact\%20Sheets\%20Library/Maize-infopak.pdf

    \bibitem{corngrowing}
    Joku.
    \emph{Corn Growing and Harvest Information}.
    Haettu \today. \\
    http://veggieharvest.com/vegetables/corn.html

    \bibitem{wheat}
    Exploring the Environment Team.
    \emph{Wheat}.
    Haettu \today. \\
    http://www.cotf.edu/ete/modules/climate/GCwheat3.html

    \bibitem{wheatfert}
    M. L. Vitosh.
    \emph{Wheat Fertility and Fertilization}.

    \bibitem{growwheat}
    GrowingAnything.com.
    \emph{How to Grow Wheat}.
    Haettu \today. \\
    http://www.growinganything.com/how-to-grow-wheat.html
        
  \end{thebibliography}

  \begin{appendices}

  \chapter{Ohjelmakoodi}
  
  \section{Juttu.java}
  
  Java-koodia.
  
  \section{ToinenJuttu.java}

  Lisää Javaa.

  \chapter{Data $\rightarrow$ kerroin}
   
  ...?

  \end{appendices}

\end{document}
